%%This is a very basic article template.
%%There is just one section and two subsections.
\documentclass{scrartcl}
\usepackage[colorlinks, urlcolor=blue]{hyperref}
\usepackage{fullpage}
\usepackage{graphics}
\usepackage{graphicx}
\usepackage{verbatim}

\title{Ohmage Installation and Administration Manual for Ubuntu 12.04 (LTS)}
\subtitle{Version 2.13-0}


\begin{document}

\maketitle

\noindent Feedback, bugs, comments, suggestions, etc, about the Ubuntu
installation packages or this document are welcome and can go to
\href{mailto:jeroen.ooms@stat.ucla.edu}{jeroen.ooms@stat.ucla.edu}.
Communication about the Ohmage software itself is easiets through github:
\href{https://github.com/cens/ohmageServer}{https://github.com/cens/ohmageServer}.


\section*{About}

The following instructions will deploy a server with:

\begin{itemize}
  \item Ubuntu 12.04
  \item Ohmage 2.13
  \item OpenCPU 0.7 (optional)
\end{itemize}

\noindent The 12.04 version of Ubuntu ships with the following
software versions of third party Ohmage dependencies:

\begin{itemize}
  \item Linux kernel 3.2.0
  \item OpenJDK 6b24
  \item MySQL 5.5.22
  \item Apache 2.2.20 (includes mod-proxy-ajp and mod-ssl)
  \item Tomcat 7.0.26
  \item Postfix 2.9.1
\end{itemize}

\noindent Please note that at this point there are no official Ubuntu builds of
Ohmage. The Ohmage installation packages for Ubuntu are kindly provided by the
\texttt{OpenCPU} project.



\section{Installation}

This section will show how to install Ohmage on Ubuntu 12.04. There are 4
Ubuntu packages: \texttt{ohmage-server}, \texttt{ohmage-viz}, 
\texttt{ohmage-standalone} and \texttt{ohmage-selfreg}. The
\texttt{ohmage-server} package is the main package, which will install the
ohmage server and web administration frontend. The \texttt{ohmage-viz} package
installs the optional vizualization server. The vizualization server can run on
a different server than \texttt{ohmage-server}. If you want to install both
Ohmage server and vizualization server on one and the same machine, this is
easiest done by installing \texttt{ohmage-standalone}. The
\texttt{ohmage-standalone} package is a very thin meta-package that will simply
install both \texttt{ohmage-server} and \texttt{ohmage-viz}, and automatically
update the Ohmage server to use the localhost vizualization server. \\

\noindent Finally installing the \texttt{ohmage-selfreg} package activates the
self-registration on the server. However, in order for self registration to
work properly, the server might need a valid domain name. For more details, see
section \ref{selfreg}.


\subsection{Installing Ubuntu}

\noindent The current build of Ohmage requires an Ubuntu 12.04 system. It can
run on any version of Ubuntu, e.g. Ubuntu Desktop, Ubuntu Server, Kubuntu, Edubuntu,
etc. If you are already have an installed system, you can skip this section. \\

\noindent The preferred way of running Ohmage is on a clean Ubuntu Server
edition. A copy of the Ubuntu Server installation disc can be obtained from the
Ubuntu download page: \\

\url{http://www.ubuntu.com/download/server} \\

\noindent If you would like to run Ohmage on an Amazon EC2 server, the best way
is to use one of the official AMI's as provided by the ubuntu team: \\

\url{http://cloud-images.ubuntu.com/precise/current/} \\

\noindent Another possibility is to install a Ohmage on a virtual Ubuntu server
inside another OS. For example, the free VMware Player is available for Windows 
and Linux, and on OSX one can use parallels to run an Ubuntu server. This way
you can install Ubuntu and Ohmage safely on top of an existing system.

\subsection{Getting the system up-to-date}

\noindent Before begin installation of Ohmage, make sure you are running Ubuntu
12.04 (Precise) by entering:

\begin{verbatim}
    cat /etc/*release
\end{verbatim}
If it turns out the system is running an older version of Ubuntu, upgrade the OS
to 12.04 first. If the system is indeed 12.04, continue by updating the software packages to the latest versions:

\begin{verbatim}
    sudo apt-get update
    sudo apt-get upgrade
\end{verbatim}
Once the system is up to date, you can begin installing Ohmage.

\subsection{Ohmage Installation}

Start by adding the \texttt{ohmage-2.13} package respository to our system:

\begin{verbatim}
    sudo apt-get install python-software-properties
    sudo add-apt-repository ppa:opencpu/ohmage-2.13
\end{verbatim}
The system will ask for confirmation on importing the public key. After the
repository has been added to the system, update the package list:

\begin{verbatim}
    sudo apt-get update
\end{verbatim}
Once this has succeeded ohmage can be installed. You have two options. To
install only the ohmage server and frontend, run:

\begin{verbatim}
    sudo apt-get install ohmage-server
\end{verbatim}
This will be sufficient to get started with Ohmage. If you want to install both
ohmage and the optional vizualization server you can install

\begin{verbatim}
    sudo apt-get install ohmage-standalone
\end{verbatim}
In the case you want to install \emph{only} the vizualization server, but not
Ohmage itself, run:
\begin{verbatim}
    sudo apt-get install ohmage-viz
\end{verbatim}
Note that all of these packages are compatible. For example, to upgrade from
\texttt{ohmage-server} to \texttt{ohmage-standalone} simply install the
latter and it will automatically make the appropriate changes. \\

\noindent Ohmage has many dependencies, and installation might take a while on
a vanilla server. During installation of \texttt{MySQL} (a dependency), the
system might ask for a password for the mysql root user. Make sure to enter a strong
password and write it down somewhere. You will not need it anymore during the
insallation though.\\

\noindent If the installation finished without any problems, it will display the
ip address of the host somewhere at the end of the output, which you can can
open in your browser and use to test the server. 

\subsection{Uninstall Ohmage}

If you want to remove Ohmage from a system you can use:

\begin{verbatim}
    sudo apt-get remove ohmage-*
    sudo apt-get autoremove
\end{verbatim}
Note that this will delete the \texttt{ohmage} database from MySQL so all
data will be lost.

\section{Administration}

The default install of Ohmage actually installs 3 sites:

\begin{itemize}
  \item Ohmage Server: \url{http://example.com/app/config/read}
  \item Ohmage Front-end: \url{http://example.com/ohmage}
  \item OpenCPU: \url{http://example.com/R} (only available with
  \texttt{ohmage-viz})
\end{itemize}
The standard user is \texttt{ohmage.admin} with password \texttt{ohmage.passwd}.
You will be prompted to change this password on first login.
By default, both http and https are enabled. However, the https is served by a
self-signed a.k.a. \emph{snakeoil} SSL certificate, so the browser will give a
warning about insecure encryption. For more info see the section \ref{ssl} of
this manual.

\subsection{Tomcat}

Ubuntu 12.04 ships with Tomat7. The Tomcat server only hosts the AJP1.3
prototcol on port 8009. Actual incoming HTTP and HTTPS are handled by Apache2
and proxied to Tomcat. To manage the Tomcat server do:

\begin{verbatim}
    sudo service tomcat7 {start | stop | restart}
\end{verbatim}
This command calls the \texttt{/etc/init.d/tomcat7} script which should usually
not be edited. Some global variables can be modified in
\texttt{/etc/default/tomcat7}. Tomcat configuration files, for example
\texttt{server.xml} are located at

\begin{verbatim}
    /etc/tomcat7/
\end{verbatim}
The tomcat7 log files \texttt{aw.log} and \texttt{catalina.out} are located
at

\begin{verbatim}
    /var/log/tomcat7/
\end{verbatim}
The \texttt{webapps} directory, hosting the \texttt{.war} files is located at

\begin{verbatim}
    /var/lib/tomcat7/webapps/
\end{verbatim}

\subsection{Apache2}

Incoming requests on port 80 (HTTP) and port 443 (HTTPS) are handled by the
Apache2 webserver. The \texttt{mod\textunderscore proxy\textunderscore ajp}
module is used to proxy requests to Tomcat server. To manage Apache2 use:

\begin{verbatim}
    sudo service apache2 {start | stop | restart}
\end{verbatim}
This command calls the \texttt{/etc/init.d/apache2} script which should usually not be edited. The main configuration file for apache2 is located at

\begin{verbatim}
    /etc/apache2/httpd.conf
\end{verbatim}
However by convention this file should rarely be edited. Custom configurations
are located at:

\begin{verbatim}
    /etc/apache2/mods-available/
    /etc/apache2/sites-available/
\end{verbatim}
These custom configurations can be activated and de-activated as follows:

\begin{verbatim}
    sudo a2enmod proxy_ajp
    sudo a2dismod proxy_ajp
    sudo a2ensite ohmage
    sudo a2dissite ohmage
\end{verbatim}
These commands create or remove symoblic inside links to available configuration
files inside the following directories:

\begin{verbatim}
    /etc/apache2/mods-enabled/
    /etc/apache2/sites-enabled/
\end{verbatim}
All files in these directories are automatically included by the main
\texttt{httpd.conf} file. The \texttt{ohmage} and \texttt{OpenCPU} sites are
defined in the following files:

\begin{verbatim}
    /etc/apache2/sites-available/ohmage
    /etc/apache2/sites-available/opencpu
\end{verbatim}
The Apache2 log files \texttt{access.log} and \texttt{error.log} are located at

\begin{verbatim}
    /var/log/apache2/
\end{verbatim}

\subsection{MySQL}

The MySQL server can be managed through:

\begin{verbatim}
    sudo service mysql {start|stop|restart}
\end{verbatim}

\noindent This command calls the \texttt{/etc/init.d/mysql} script which should
usually not be edited. Some global settings can be modified in
\texttt{/etc/mysql/debian-start} and \texttt{/etc/mysql/my.cnf}. In general, it
should not be required to manually enter mysql for using Ohmage. But if for some
reason you want to, you can connect to the mysql server using:

\begin{verbatim}
    mysql -u ohmage -p
\end{verbatim}
The password is \texttt{\&!sickly} and all ohmage data is stored in database
\texttt{ohmage}.

\subsection{OpenCPU (part of ohmage-viz)}

OpenCPU is used by the Ohmage-frontend to offer visualizations for data
exploration. If you do not plan on using data vizualization, or use an
external vizualization server, opencpu can be disabled:

\begin{verbatim}
    sudo a2dissite opencpu
\end{verbatim}
To change the vizualization server used by Ohmage, connect to MySQL and issue
the following command:

\begin{verbatim}
    use ohmage;
    update preference set p_value = "http://viz.example.com/R/call/Mobilize/"
    where p_key = "visualization_server_address";
\end{verbatim}
Where the server url is replaced by the appropriate viz server. To
restore it to the default value, run:

\begin{verbatim}
    use ohmage;
    update preference set p_value = "http://127.0.0.1/R/call/Mobilize/" where
    p_key = "visualization_server_address";
\end{verbatim}

\subsection{SSL certificate}
\label{ssl}

By default, Apache2 uses self signed a.k.a. snakeoil certificates. This is
convenient for development servers, but in a production setting these
should be replaced by SSL certificates signed by an official Certificate Authority. \\

\noindent The https configurations and locations of the certificates are defined
in

\begin{verbatim}
    /etc/apache2/sites-available/default-ssl
\end{verbatim}
This file also contains detailed comments with configuration instructions.

\subsection{Self registration}
\label{selfreg}

Ohmage supports option self registration. This means that users can register an 
account for themselves without any help from an administrator. The self
registration module can be installed as follows:

\begin{verbatim}
    sudo apt-get install ohmage-selfreg
\end{verbatim}

\noindent As part of the self registration process, a user will receive an email
with a confirmation code, and a link back to the server. In order for this to
work properly, the server needs a valid hostname. The hostname of the server is
defined in this file:

\begin{verbatim}
    /etc/hostname
\end{verbatim}

\noindent The link that is included in the confirmation email that self
registered user receive, is determined by this file, so make sure it contains a
proper hostname, and not e.g. \texttt{localhost} or some internal name. \\

\noindent The self registration depends on a properly functioning
SMTP server on the system, either Postfix or Sendmail. These will automatically
be installed when installing \texttt{ohmage-selfreg}. During the installation of
Postfix you might be propted for the hostname of your server. Again, make sure
that you enter a valid hostname here that can be reached through the internet.

\subsubsection{Important: Reverse DNS and Spam Detection}

Because spam is a big problem these days, most email providers tend to flag
emails that have been send from anonymous SMTP servers as spam. As a result,
the self registration confirmation emails might end up in their spam-folder or
junkmail. In order to minimize the chance that emails from Ohmage end up in spam
filters, it is highly recommended to use a domain that you actually purchased,
and not just the hostname of the machine that your ISP/hosting partner provided.
Furthermore it is important that the \emph{\textbf{reverse DNS}} of the server
to points back to this same domain name. Setting the reverse DNS is a process
that only your hosting provider can do for you. Most providers require you to
request this manually, for example, on EC2 you have to fill out this form: \\

{\footnotesize\url{https://aws-portal.amazon.com/gp/aws/html-forms-controller/contactus/ec2-email-limit-rdns-request}}
\\



\noindent In order to test if the the DNS and Reverse DNS are working properly,
you can use a command like \texttt{nslookup} on Linux or \texttt{tracert} on
Windows. Alternatively you can use a free web tool to do the lookup for you,
for example \url{http://www.dnsgoodies.com/}.

 



\subsection{Other Ohmage files}

Photos, videos and documents uploaded by users are stored in

\begin{verbatim}
    /opt/ohmage/userdata/images/
    /opt/ohmage/userdata/documents/
    /opt/ohmage/userdata/videos/
\end{verbatim}
Other than this, \texttt{/opt/ohmage/} contains a directory with log files:

\begin{verbatim}
    /opt/ohmage/logs
\end{verbatim}

\noindent Note that these are only high level ohmage logs. If there are problems
with the web server or database itsefl, these might appear in the tomcat logs. 

\section{Clients}

Currently there are 3 clients for the Ohmage server system. These are:

\begin{itemize}
  \item The Ohmage Android App.
  \item The Ohmage FrontEnd.
  \item The Ohmage R package.
\end{itemize}
Below a brief description of these clients.

\subsection{The Ohmage Android App}

The Ohmage Android 'app' is the application on the mobile phone that can be used
to fill out surveys and upload survey-responses to the server. As it currently
stands, the server-url is hardcoded in the app and therefore the app has to be
built from source. Figure \ref{fig:phone} shows a screenshot of the phone app running on
an Android 2.2 device. \\

\begin{figure}[h!]
\begin{center}
\includegraphics[width=4cm]{app.png}
\caption{A screenshot of the Android app.}
\label{fig:phone}
\end{center}
\end{figure}

\noindent The source code and instructions on how to build the app are publicly
available on: \\

\url{https://github.com/cens/ohmagePhone} \\

\subsection{The Ohmage FrontEnd}

The Ohmage FrontEnd is an administrative web application to be used on a regular
browser by both users and administrators of Ohmage. The application is
automatically installed when installing the server using instructions above and
available through: \url{http://example.com/ohmage}. Source code and
development of the FrontEnd is publicly available on github at
\url{https://github.com/cens/ohmageFrontEnd}. Figure \ref{fig:frontend} shows a
screenshot of the FrontEnd homepage after logging in. \\

\noindent The FrontEnd is a convenient client to review, share and explore data,
add/remove users, classes, campaigns, perform administrative tasks, etc. The
frontend can be build with some custom skinning options. The screenshot shows a build of
Ohmage with the default theme.

\begin{figure}[h!]
\begin{center}
\includegraphics[width=15cm]{frontend.png}
\caption{A screenshot of the FrontEnd homepage.}
\label{fig:frontend}
\end{center}
\end{figure}

\subsection{The Ohmage \texttt{R} package}

The Ohmage R package is an Ohmage client for R. It depends on other R packages
like RCurl, XML and RJSONIO to do it's work. The package is mostly a convenient
way to grab data from Ohmage and turn it into a data frame in R. Package and
documentation are available from CRAN:
\url{http://cran.r-project.org/web/packages/Ohmage}. Below a code snippet to
illustrate the functionality of the package.

\begin{verbatim}
  library(Ohmage);
  oh.login("ohmage.admin", "mypassword", "https://myserver.com/app");
  campaigns <- oh.campaign.read();
  mydata <- oh.survey_response.read("urn:campaign:myschool:food");
\end{verbatim}



\end{document}
